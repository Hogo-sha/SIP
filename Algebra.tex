\documentclass{article}

\usepackage[margin=1in]{geometry}
\usepackage{amsmath,amsthm,amssymb}

\newcommand{\R}{\mathbb{R}}
\newcommand{\Z}{\mathbb{Z}}
\newcommand{\N}{\mathbb{N}}
\newcommand{\Q}{\mathbb{Q}}
\newcommand{\C}{\mathbb{C}}

\newenvironment{theorem}[2][Theorem]{\begin{trivlist}
\item[\hskip \labelsep {\bfseries #1}\hskip \labelsep {\bfseries #2.}]}{\end{trivlist}}
\newenvironment{lemma}[2][Lemma]{\begin{trivlist}
\item[\hskip \labelsep {\bfseries #1}\hskip \labelsep {\bfseries #2.}]}{\end{trivlist}}
\newenvironment{exercise}[2][Exercise]{\begin{trivlist}
\item[\hskip \labelsep {\bfseries #1}\hskip \labelsep {\bfseries #2.}]}{\end{trivlist}}
\newenvironment{problem}[2][Problem]{\begin{trivlist}
\item[\hskip \labelsep {\bfseries #1}\hskip \labelsep {\bfseries #2.}]}{\end{trivlist}}
\newenvironment{question}[2][Question]{\begin{trivlist}
\item[\hskip \labelsep {\bfseries #1}\hskip \labelsep {\bfseries #2.}]}{\end{trivlist}}
\newenvironment{corollary}[2][Corollary]{\begin{trivlist}
\item[\hskip \labelsep {\bfseries #1}\hskip \labelsep {\bfseries #2.}]}{\end{trivlist}}

\newenvironment{solution}{\begin{proof}[Solution]}{\end{proof}}

\begin{document}

% ------------------------------------------ %
%                 START HERE                 %
% ------------------------------------------ %

\title{Ms. Peters Algebra} % Replace X with the appropriate number
\author{Heather Peters } % Replace "Author's Name" with your name

\maketitle

% -----------------------------------------------------
% The following two environments (theorem, proof) are
% where you will enter the statement and proof of your
% first problem for this assignment.
%
% In the theorem environment, you can replace the word
% "theorem" in the \begin and \end commands with
% "exercise", "problem", "lemma", etc., depending on
% what you are submitting. Replace the "x.yz" with the
% appropriate number for your problem.
%
% If your problem does not involve a formal proof, you
% can change the word "proof" in the \begin and \end
% commands with "solution".
% -----------------------------------------------------

\begin{problem}{1}
The cost of petrol rises by 2 cents a liter. last week a man bought 20 liters at the old price. This week he bought 10 liters at the new price. Altogether, the petrol costs $9.20. What was the old price for 1 liter?
\end{problem}

\begin{proof}

\end{proof}

% -----------------------------------------------------
% Second problem
% -----------------------------------------------------

\vspace{0.25in} % This just adds some space between problems.

\begin{problem}{2}
One ounce of solution X contains only ingredients a and b in a ratio of 2:3. One ounce of solution Y contains only ingredients a and b in a ratio of 1:2. If solution Z is created by mixing solutions X and Y in a ratio of 3:11, then 2520 ounces of solution Z contains how many ounces of a?
\end{problem}

\begin{proof}

\end{proof}

% -----------------------------------------------------
% Third problem
% -----------------------------------------------------

\vspace{0.25in} % This adds some space between problems.

\begin{problem}{3}
A commercial airplane flying with a speed of 700 mi/h is detected 1000 miles away with a radar. Half an hour later an interceptor plane flying with a speed of 800 mi/h is dispatched. How long will it take the interceptor plane to meet with the other plane?
\end{problem}

\begin{proof}

\end{proof}

% -----------------------------------------------------
% Fourth problem
% -----------------------------------------------------

\vspace{0.25in} % This adds some space between problems.

\begin{problem}{4}
You are raising money for a charity. Someone made a fixed donation of 500. Then, you require each participant to make a pledge of 25 dollars. What is the minimum amount of money raised if there are 224 participants.
\end{problem}

\begin{proof}

\end{proof}

% -----------------------------------------------------
% Fifth problem
% -----------------------------------------------------

\vspace{0.25in} % This adds some space between problems.

\begin{problem}{5}
 The ratio of girls to boys in class is 9 to 7 and there are 80 students in the class. How many girls are in the class?
\end{problem}

\begin{proof}

\end{proof}

% -----------------------------------------------------
% Sixth problem
% -----------------------------------------------------

\vspace{0.25in} % This adds some space between problems.

\begin{problem}{6}
Vera and Vikki are sisters. Vera is 4 years old and Vikki is 13 years old. What age will each sister be when Vikki is twice as old as Vera?
\end{problem}

\begin{proof}

\end{proof}

% -----------------------------------------------------
% Seventh problem
% -----------------------------------------------------

\vspace{0.25in} % This adds some space between problems.

\begin{problem}{7}
The sum of two positive numbers is 4 and the sum of their squares is 28. What are the two numbers?
\end{problem}

\begin{proof}

\end{proof}

% -----------------------------------------------------
% Eighth problem
% -----------------------------------------------------

\vspace{0.25in} % This adds some space between problems.

\begin{problem}{8}
A can do a work in 14 days and working together A and B can do the same work in 10 days. In what time can B alone do the work?
\end{problem}

\begin{proof}

\end{proof}

% -----------------------------------------------------
% Ninth problem
% -----------------------------------------------------

\vspace{0.25in} % This adds some space between problems.

\begin{problem}{9}
Jenna and her friend, Khalil, are having a contest to see who can save the most money. Jenna has already saved $110 and every week she saves an additional $20. Khalil has already saved $80 and every week he saves an additional $25. Let x represent the number of weeks and y represent the total amount of money saved. Determine in how many weeks Jenna and Khalil will have the same amount of money.
\end{problem}

\begin{proof}

\end{proof}

% -----------------------------------------------------
% Tenth problem
% -----------------------------------------------------

\vspace{0.25in} % This adds some space between problems.

\begin{problem}{10}
There are 40 pigs and chickens in a farmyard. Joseph counted 100 legs in all. How many pigs and how many chickens are there?
\end{problem}

\begin{proof}

\end{proof}


% -----------------------------------------------
% Ignore everything that appears below this.
% -----------------------------------------------

\end{document}
